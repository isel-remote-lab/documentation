%
% Chapter 2
%
\chapter{Placement} \label{cap:placement}

This chapter is organized into two sections, where we describe related work and some systems similar to the one developed in this project.

\section{Related Work}
In recent years, several initiatives have emerged to provide remote access to laboratory resources, especially in the context of higher education and research. Projects such as MIT's iLab and LabShare have demonstrated the feasibility and benefits of remote laboratories, enabling students and researchers to conduct experiments from anywhere in the world. These platforms typically focus on providing secure access, scheduling, and integration with a variety of laboratory equipment. The literature highlights the importance of usability, scalability, and security in the design of such systems, as well as the challenges associated with real-time interaction and hardware integration.

\section{Similar Systems}
There are several systems that offer functionalities similar to those of the Remote Lab project. For example, the iLab Shared Architecture (ISA) provides a framework for sharing laboratory equipment over the internet, supporting both batch and interactive experiments. LabShare is another notable example, offering a collaborative platform for remote experimentation and resource sharing among institutions. Other systems, such as WebLab-Deusto and VISIR, focus on specific domains like electronics and instrumentation, providing specialized interfaces and tools for remote experimentation. These systems serve as valuable references for the development of the Remote Lab platform, informing decisions related to architecture, user experience, and integration with laboratory hardware.
