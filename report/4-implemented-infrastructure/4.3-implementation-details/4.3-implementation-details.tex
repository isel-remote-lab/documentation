\section{Implementation Details}

Key implementation decisions and details include:

\begin{itemize}
    \item \textbf{Containerization:} All major components (frontend, backend, database) are containerized using Docker, ensuring consistent environments across development and production.
    \item \textbf{Orchestration:} Docker Compose is used to manage multi-container deployments, networking, and environment configuration.
    \item \textbf{Backend:} The backend uses JDBI for type-safe database access and is configured via environment variables for flexibility and security.
    \item \textbf{Frontend:} The frontend is built with Next.js, providing a modern, responsive interface and integrating with the backend via RESTful APIs.
    \item \textbf{Authentication:} Microsoft OAuth (via NextAuth) is used for secure authentication, enabling university login. The system is designed to be flexible, so other OAuth providers or even an internal login mechanism could be used if required.
    \item \textbf{Role-based Access Control:} The system enforces permissions based on user roles, ensuring secure and appropriate access to resources.
    \item \textbf{Hardware Abstraction:} The backend abstracts hardware-specific details, allowing for easy extension to new laboratory equipment.
    \item \textbf{Automation:} The \texttt{start.sh} script automates the bootstrap process, starting all necessary services with a single command.
\end{itemize}

\subsection*{Role-Based Access Control (RBAC)}

The system implements Role-Based Access Control (RBAC) to ensure that users have access only to the resources and actions appropriate for their role. Each authenticated user is assigned a role, such as student, professor, or administrator, which determines their permissions within the platform. 

Roles are enforced both on the backend and frontend. On the backend, endpoints and business logic check the user's role before allowing access to sensitive operations, such as managing laboratory sessions, accessing administrative features, or modifying user data. On the frontend, the user interface dynamically adapts to the user's role, displaying only the features and options relevant to their permissions.

The main roles in the system are:
\begin{itemize}
    \item \textbf{Student:} Can view and book laboratory sessions, access their own session history, and interact with laboratory equipment during their reserved times. Students have access only to features relevant to their participation in laboratory activities.
    \item \textbf{Professor:} In addition to all student permissions, professors can create and manage laboratory sessions, view and manage student participation, and access additional data and reports related to their classes or laboratories.
    \item \textbf{Administrator:} Has full access to all system features, including user management, system configuration, and oversight of all laboratory sessions and resources. Administrators can manage roles, permissions, and perform maintenance or troubleshooting tasks across the platform.
\end{itemize}

This approach provides a secure and flexible way to manage access, making it easy to introduce new roles or adjust permissions as the platform evolves. The RBAC system is central to maintaining the integrity and security of the Remote Lab environment.

These choices ensure the platform is robust, extensible, and easy to deploy or develop locally.

In addition, the web application allows users to view and interact with the platform as if they had a lower role than their own. This feature is particularly useful for testing, support, and understanding the user experience from different perspectives. For example, an administrator or professor can switch to a student view to verify permissions, troubleshoot issues, or provide guidance, without needing to log in as a different user. 