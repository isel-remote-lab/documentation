\section{Deployment}

The deployment process for the Remote Lab platform is designed to be straightforward, secure, and reproducible, leveraging modern DevOps practices and containerization technologies.

\subsection{Containerization and Orchestration}
All major components of the platform—including the backend (api), frontend (website), and database—are containerized using Docker. This ensures consistency across development, testing, and production environments. Docker Compose is used to orchestrate multi-container deployments, manage networking between services, and handle environment-specific configurations.

\subsection{Environment Configuration and Secrets}
Sensitive configuration files and environment variables required for deployment are managed in the \textbf{private/} submodule. This submodule contains the necessary secrets, such as API keys, database credentials, and authentication settings, tailored to the specific requirements of the platform. Access to this submodule is restricted to authorized team members, ensuring the security of confidential information.

\subsection{Automation with start.sh}

To further streamline the deployment process, the platform provides a \texttt{start.sh} script located at the root of the repository. This script automates the bootstrap process by orchestrating the initialization of all required services and dependencies with a single command. It handles tasks such as building Docker images, starting containers using Docker Compose, and ensuring that environment variables and configuration files are correctly loaded from the \texttt{private/} submodule.

The \texttt{start.sh} script also supports several flags to customize the deployment process, such as selecting the environment (development or production), starting only the API, enabling Cloudflare tunneling, or switching branches. These options make it easy to adapt the deployment to different scenarios with simple command-line arguments.

\subsection{Deployment Steps}
\begin{enumerate}
    \item \textbf{Clone the Repository and Submodules:} Clone the main repository and initialize all submodules, including \texttt{private/}, to ensure all components and configurations are available.
    \item \textbf{Configure Environment Variables:} Ensure that all required environment variables and secret files are present in the appropriate locations, as provided by the \texttt{private/} submodule.
    \item \textbf{Build and Start Services:} Use the provided \texttt{docker-compose.yml} file to build and start all services with a single command (e.g., \texttt{docker compose up --build}).
    \item \textbf{Access the Platform:} Once all containers are running, the platform can be accessed via the configured web address. Nginx is used as a reverse proxy to route traffic securely to the appropriate services.
\end{enumerate}

\subsection{Local and Production Deployment}
The deployment process is designed to be nearly identical for both local development and production environments. Developers can run the entire stack locally using Docker Compose, mirroring the production setup. For production, additional considerations such as SSL certificates, domain configuration, and scaling may be applied, but the core process remains the same.

This approach ensures that deployments are reliable, repeatable, and secure, minimizing the risk of configuration drift and simplifying both initial setup and ongoing maintenance.
