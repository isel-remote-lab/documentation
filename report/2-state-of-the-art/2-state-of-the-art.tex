%
% Chapter 2
%
\chapter{State of the Art} \label{cap:state-of-the-art}

In recent years, several initiatives have emerged to provide remote access to laboratory resources, especially in the context of higher education and research. Projects such as MIT's iLab~\cite{ilab} and LabShare~\cite{labshare} have demonstrated the feasibility and benefits of remote laboratories, enabling students and researchers to conduct experiments from anywhere in the world. These platforms typically focus on providing secure access, scheduling, and integration with a variety of laboratory equipment. The literature highlights the importance of usability, scalability, and security in the design of such systems, as well as the challenges associated with real-time interaction and hardware integration.

There are several systems that offer functionalities similar to those of the Remote Lab project. For example, the iLab Shared Architecture (ISA), developed by MIT, is no longer operational and is not available for public access~\cite{ilab}. LabShare is another notable example, but it is currently also unavailable~\cite{labshare}. Other systems, such as WebLab-Deusto, focus on specific domains like electronics and instrumentation, providing specialized interfaces and tools for remote experimentation~\cite{weblabdeusto}. These systems serve as valuable references for the development of the Remote Lab platform, informing decisions related to architecture, user experience, and integration with laboratory hardware.
