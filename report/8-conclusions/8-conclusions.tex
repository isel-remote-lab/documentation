\chapter{Conclusions}
\label{cap:conclusions}

This report has presented the design, implementation, and validation of a comprehensive platform for remote laboratory access, addressing the growing need for flexible and accessible laboratory resources in educational and research environments. The solution successfully bridges the gap between traditional hands-on experimentation and the requirements of modern remote learning and collaboration.

The developed platform successfully fulfills all primary objectives established at the project's inception. The implementation of a robust web \acs{api} provides comprehensive management capabilities for users, laboratories, and hardware resources. The intuitive web application interface ensures accessibility for users with varying technical backgrounds, while the secure authentication and authorization mechanisms, coupled with role-based access control, guarantee system integrity and appropriate resource allocation.

The well-designed database system ensures reliable data persistence, supporting the platform's scalability requirements. Furthermore, the implementation of remote manipulation capabilities of laboratory devices enables real-time interaction with experimental equipment, maintaining the essential hands-on experience that characterizes quality laboratory work.

\section{Limitations and Future Work}
\label{sec:limitations_future_work}

While the current implementation successfully addresses the primary objectives, several areas present opportunities for future enhancement. The expansion of the testing framework to include comprehensive load testing will provide valuable insights into system performance under high-usage conditions. This analysis will inform optimization strategies and guide infrastructure scaling decisions.

Future development efforts should focus on expanding hardware compatibility to support a broader range of laboratory equipment types. Additionally, the implementation of advanced monitoring and analytics capabilities could provide valuable usage insights and facilitate resource optimization. The integration of collaborative features, such as shared experimentation sessions and real-time collaboration tools, would further enhance the platform's educational value.

Furthermore, the development of mobile applications and the enhancement of accessibility features would broaden the platform's reach and usability across diverse user populations.