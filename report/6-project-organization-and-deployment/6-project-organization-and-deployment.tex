\chapter{Project Organization and Deployment} \label{cap:project_org_deployment}

\section{Project Structure}

The Remote Lab project is organized into several main directories, most of which are managed as GitHub submodules. This modular approach enables independent development, versioning, and access control for each core component, supporting both scalability and security. Submodules also facilitate collaboration among different teams and ensure that sensitive information is handled appropriately.

The main submodules and directories are:

\begin{itemize}
    \item \textbf{api/} – Contains the backend source code, implemented in Kotlin with Spring Boot. Responsible for business logic, user management, and laboratory session control.
    \item \textbf{db/} – Includes database scripts, supporting the system's persistence layer.
    \item \textbf{docs/} – Stores project documentation, including technical reports, user guides, and architectural diagrams.
    \item \textbf{img/} – Contains project images, such as diagrams, screenshots, and other visual assets.
    \item \textbf{nginx/} – Provides NGINX configuration files for reverse proxying, load balancing, and secure access to backend services~\cite{nginx-docs}.
    \item \textbf{private/} – Dedicated to sensitive files and configurations, such as environment variables and secrets. This submodule is not included directly in the main repository, ensuring that only authorized members have access to confidential information like \ac{api} keys and external service credentials.
    \item \textbf{website/} – Holds the frontend web application, built with Next.js (React)~\cite{nextjs-docs}. Provides the user interface for laboratory access, scheduling, and management.
    \item \textbf{wiki/} – Stores the GitHub Wiki pages, including project documentation, deployment instructions, and other relevant information.
\end{itemize}

This structure allows for independent development, testing, and deployment of each component, while best practices such as containerization and secure secret management ensure the project is robust and ready for collaborative development and future expansion.

\section{Deployment} \label{sec:deployment}

The deployment process for the Remote Lab platform is designed to be straightforward, secure, and reproducible, leveraging modern DevOps practices and containerization technologies~\cite{docker-docs,docker-compose-docs}.

\subsection{Containerization and Orchestration} \label{subsec:containerization}

All major components—including the backend (api), frontend (website), and database—are containerized using Docker~\cite{docker-docs}. This guarantees consistency across development, testing, and production environments. Docker Compose orchestrates multi-container deployments, manages networking between services, and handles environment-specific configurations~\cite{docker-compose-docs}.

\subsection{Environment Configuration and Secrets} \label{subsec:env_config}

Sensitive configuration files and environment variables are managed in the \textbf{private/} submodule. This submodule contains secrets such as \ac{api} keys, database credentials, and authentication settings, tailored to the platform's requirements. Access is restricted to authorized team members, ensuring the security of confidential information.

\subsection{Automation with start.sh} \label{subsec:automation_startsh}

To streamline deployment, the platform provides a \texttt{start.sh} script at the repository root. This script automates the initialization of all required services and dependencies with a single command. It builds Docker images, starts containers using Docker Compose, and ensures that environment variables and configuration files are correctly loaded from the \texttt{private/} submodule.

The \texttt{start.sh} script also supports several flags to customize the deployment process, such as selecting the environment (development or production), starting only the api, enabling Cloudflare tunneling, or switching branches. These options make it easy to adapt deployment to different scenarios with simple command-line arguments.

\subsection{Cloudflare Tunneling} \label{subsec:cloudflare_tunneling}

To facilitate secure remote access to development or demonstration environments, Cloudflare Tunneling is used~\cite{cloudflare-tunnel-docs}. This solution exposes local services to the internet without requiring firewall changes, port forwarding, or public \ac{ip}s. Cloudflare Tunnel creates a secure connection between the local machine and a public endpoint managed by Cloudflare, routing external traffic in an encrypted manner to the internal environment using the \ac{http} or \ac{https} protocols. This is especially useful for:

\begin{itemize}
    \item Allowing team members, professors, or evaluators to access development instances remotely.
    \item Demonstrating the system without deploying to a production environment.
    \item Testing external integrations or federated authentication in controlled environments.
\end{itemize}

In this project, Cloudflare Tunnel is integrated into the automation workflow via \texttt{start.sh} and Docker Compose. By using the appropriate flag (\texttt{cloudflare} or \texttt{c}) in the startup script, the tunnel service is automatically started alongside the other containers, making remote access simple and secure.

This approach reduces operational complexity, increases security, and streamlines the system's development and validation cycle.

\subsection{Nginx as Reverse Proxy} \label{subsec:nginx_reverse_proxy}

In the current implementation, NGINX acts as a reverse proxy to route \ac{http} requests to both the frontend (website) and backend (API) services~\cite{nginx-docs}. This allows the web application and the API to be accessed through a single entry point, even though they run in separate containers and listen on different internal ports. NGINX transparently proxies requests from the frontend to the backend, simplifying client-side configuration and enabling seamless communication between modules.

At this stage, NGINX is not configured for load balancing or native HTTPS termination. All traffic is handled over HTTP within the Docker network, and any external HTTPS is managed by tunneling solutions such as Cloudflare when needed.

The NGINX configuration files are located in the \texttt{nginx/} directory and are included in the Docker Compose setup, ensuring that the reverse proxy is automatically started and configured alongside the other services during deployment.

\subsection{Deployment Steps} \label{subsec:deployment_steps}

\begin{enumerate}
    \item \textbf{Clone the Repository and Submodules:} Clone the main repository and initialize all submodules, including \texttt{private/}, to ensure all components and configurations are available.
    \item \textbf{Configure Environment Variables:} Ensure that all required environment variables and secret files are present in the appropriate locations, as provided by the \texttt{private/} submodule.
    \item \textbf{Build and Start Services:} Use the provided \texttt{docker-compose.yml} file to build and start all services with a single command (e.g., \texttt{docker compose up --build}), or simply run the \texttt{start.sh} script at the project root, which automates the entire process including building images, starting containers, and loading environment variables and secrets.
    \item \textbf{Access the Platform:} Once all containers are running, the platform can be accessed via the configured web address. NGINX is used as a reverse proxy to route traffic securely to the appropriate services~\cite{nginx-docs}.
\end{enumerate}

\section{Build and CI/CD} \label{sec:build_cicd}

\begin{itemize}
    \item \textbf{Gradle:} Used for building and managing backend dependencies.
    \item \textbf{NPM:} Used for frontend dependency management and builds.
    \item \textbf{Dockerfiles:} Multi-stage builds are used for both backend and frontend to optimize image size and security with Docker.
    \item \textbf{GitHub Actions:} (If applicable) Used for continuous integration and automated builds as part of the \ac{cicd} pipeline~\cite{github-actions-docs}.
\end{itemize}

\section{Summary}

The implemented infrastructure leverages modern web technologies, containerization, and modular design to provide a robust, scalable, and maintainable platform for remote laboratory access~\cite{nextjs-docs,docker-docs}. The deployment process is automated, secure, and ready for future growth, ensuring that the platform can evolve to meet new requirements and increased demand.