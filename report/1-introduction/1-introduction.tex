%
% Chapter 1
%
\chapter{Introduction} \label{cap:intro}

%
% Secção 1.1
%
\section{Context and Motivation} \label{sec11}
In the recent years, the need for remote access to laboratory resources has grown significantly, driven by the expansion of online education, research collaboration, and the increasing complexity of experimental setups. Traditional laboratories often require physical presence, which can limit accessibility and flexibility for students, researchers, and professionals.
The \textbf{Remote Lab} here is proposed to address these challenges by providing a platform that enables secure, efficient, and user-friendly remote access to laboratory equipment and resources.

\section{State of the Art}

In recent years, several initiatives have emerged to provide remote access to laboratory resources, especially in the context of higher education and research. Projects such as \acs{mit}'s iLab~\cite{ilab} and LabShare~\cite{labshare} have demonstrated the feasibility and benefits of remote laboratories, enabling students and researchers to conduct experiments from anywhere in the world. These platforms typically focus on providing secure access, scheduling, and integration with a variety of laboratory equipment. The literature highlights the importance of usability, scalability, and security in the design of such systems, as well as the challenges associated with real-time interaction and hardware integration.

There are several systems that offer functionalities similar to those of the Remote Lab project. For example, the \acs{isa}, developed by \acs{mit}, is no longer operational and is not available for public access~\cite{ilab}. LabShare is another notable example, but it is currently also unavailable~\cite{labshare}. Other systems, such as WebLab-Deusto, focus on specific domains like electronics and instrumentation, providing specialized interfaces and tools for remote experimentation~\cite{weblabdeusto}. These systems serve as valuable references for the development of the Remote Lab platform, informing decisions related to architecture, user experience, and integration with laboratory hardware.

%
% Secção 1.2
%
\section{Objectives} \label{sec12}
The main objectives of the Remote Lab project are:
\begin{itemize}
    \item To design and implement a scalable platform for remote laboratory access.
    \item To ensure secure authentication and authorization for different user roles.
    \item To provide an intuitive user interface for managing and scheduling laboratory sessions.
    \item To support integration with various types of laboratory hardware.
\end{itemize}

This project focuses on the development of the core platform, including backend services, user management, and basic hardware integration. Advanced features such as real-time data analytics, support for a wide range of laboratory devices, and extensive reporting capabilities are considered out of scope for the current phase.

The project follows a modular and iterative development approach, leveraging modern software engineering practices. The backend is implemented using Kotlin and follows a layered architecture, while the frontend is developed with Next.js to provide a responsive and accessible user experience.

%
% Secção 1.3
%
\section{Structure of the Document} \label{sec13}
The structure of this report is as follows: Chapter 2 presents the related work, with a general overview of the existing solutions and technologies; Chapter 3 describes the system architecture; Chapter 4 details the main components and their interactions; Chapter 5 evaluates the system's performance and usability; and Chapter 6 presents the conclusions and future work perspectives.