% Classe do documento e parâmetros gerais.
\documentclass[a4paper,openright,twoside,11pt]{report}

% Packages utilizadas e respetivos parâmetros.
\usepackage[utf8]{inputenc}
\usepackage[english]{babel}
\addto{\captionsenglish}{\renewcommand{\bibname}{References}}
\addto{\captionsenglish}{\renewcommand{\contentsname}{Contents}}
\addto{\captionsenglish}{\renewcommand{\appendixname}{Appendix}}

\usepackage{lipsum} % gerador de texto
\usepackage{graphicx}
\usepackage{url}
\usepackage[Algoritmo]{algorithm}
\usepackage{algorithmicx}
\usepackage{algpseudocode}
\usepackage{hyperref}
\usepackage{listings}
\usepackage[
  list-style = table,         % lista em forma de tabela
  pages       = multiple      % recolhe todas as páginas de uso
]{acro}
\acsetup{make-links}
\usepackage{svg}
\renewcommand{\algorithmicrequire}{\textbf{Dados: }}
\renewcommand{\algorithmicensure}{\textbf{Resultado: }}

% Definições das dimensões das páginas
\setlength{\textheight}{24.00cm}
\setlength{\textwidth}{16cm}
\setlength{\topmargin}{0.35cm}
\setlength{\headheight}{0cm}
\setlength{\headsep}{0cm}
\setlength{\oddsidemargin}{0.25cm}
\setlength{\evensidemargin}{0.25cm}

%\renewcommand{\baselinestretch}{1}

% Declaração de acrónimos
\DeclareAcronym{bsc}{
  short=BSc,
  long=Bachelor of Science
}
\DeclareAcronym{api}{
  short=API,
  long=Aplication Programming Interface
}
\DeclareAcronym{er model}{
  short=ER Model,
  long=Entity-Relationship Model
}
\DeclareAcronym{isa}{
  short=ISA,
  long=iLab Shared Architecture
}
\DeclareAcronym{mit}{
  short=MIT,
  long=Massachusetts Institute of Technology
}
\DeclareAcronym{uri}{
  short=URI,
  long=Uniform Resource Identifier
}
\DeclareAcronym{http}{
  short=HTTP,
  long=HyperText Transfer Protocol
}
\DeclareAcronym{rbac}{
  short=RBAC,
  long=Role-Based Access Control
}

\begin{document}
\pagenumbering{roman}
\thispagestyle{empty}

%=================== Página de título ===================
\begin{titlepage}
  \begin{center}
    \begin{minipage}[l]{\textwidth}
        \hspace{-15mm}\resizebox{85mm}{!}{\includegraphics{../../img/figures/logoISELnew2.png}}\\
    \end{minipage}\\[10mm]
    {\large \textbf{Department of Electronical Engineering, Telecommunications and Computers}\\[1.5cm]}
    {\Huge \textbf{Remote Lab}\\[2cm]}    
    {\large \begin{tabular}{l}
      50565: Ângelo Filipe Maia Azevedo \href{mailto:a50565@alunos.isel.pt}{(a50565@alunos.isel.pt)}\\
      50539: António Miguel Alves \href{mailto:a50539@alunos.isel.pt}{(a50539@alunos.isel.pt)}\\
    \end{tabular}\\[3cm] }
    {\large Report for Project and Seminar Class
    
    of Computer Science and Computer Engineering \acs{bsc}\\[3cm]}
    {\large Advisor: Prof. Pedro Miguens Matutino}   
    \vfill
    {\large \textbf{June 2025}}   
  \end{center}
\end{titlepage}

\baselineskip 18pt % line spacing: 12pt for single, 18pt for 1 1/2, and 24pt for double spacing

\newpage
\thispagestyle{empty}
% Fim da contracapa

% Página com identificação completa (número e nome) e assinaturas do(s) estudante(s) e do(s) orientador(es)
\cleardoublepage
\begin{center}
\textsc{\LARGE Lisbon School of Engineering}\\[15mm]

{\Large \bf  Remote Lab}\\[20mm]

\begin{tabular}{rl}
  50565  & Ângelo Filipe Maia Azevedo\\[10mm]
           & \rule{75mm}{0.5pt}\\[5mm]

  50539  & António Miguel Alves\\[10mm]
           & \rule{75mm}{0.5pt}\\[5mm]

  Advisor: & Prof. Pedro Miguens Matutino\\[10mm]
      & \rule{75mm}{0.5pt}\\[5mm]
\end{tabular}\\[5cm]
{\large Report for Project and Seminar Class of Computer Science and Computer Engineering \acs{bsc}\\[3cm]}
June 2025\\
\end{center}

% Página de resumo em Inglês
\cleardoublepage
\chapter*{Abstract}
The design, development, implementation, and validation of digital systems require, in addition 
to simulators, the use of hardware for verification of their implementation in real devices. 
However, access to these real devices is sometimes restricted, not being available 24/7. 
In the current teaching paradigm where face-to-face time is reduced and remote and autonomous work 
is increased, it is necessary to create alternatives to the usual model.

The Remote Lab project aims to provide an online laboratory with access to remote hardware.
This remote workbench consists of a web application running on an embedded system. The web application, accessed through a website, 
aims to provide a dashboard where users can join a laboratory. This
is where users can control the remote hardware. A hierarchy system will be implemented to
provide different roles, each with their own permissions relative to how users can browse the
information provided by the web application.

This project will implement the infrastructure to support the configuration, manipulation and visualization
of remote hardware. Based on an architecture with back-end (database and Web \acs{api}) and front-end (Web App, with a dashboard). 

\cleardoublepage
\chapter*{Resumo}
A conceção, desenvolvimento, implementação, e por fim a validação de sistemas digitais 
requerem para além dos simuladores, a utilização de hardware para uma verificação da sua 
concretização em dispositivos reais. No entanto, o acesso a esses dispositivos reais é por 
vezes restrito, não estando acessíveis 24h/7. No atual paradigma de ensino em que se reduz 
o tempo de contacto presencial, aumentando-se o trabalho remoto e autónomo, é necessário criar alternativas 
ao modelo habitual. 

O projeto Remote Lab tem como objetivo fornecer um laboratório online com acesso a hardware remoto.
Este laboratório consiste numa aplicação web executada num sistema embebido. A aplicação web, acedida 
através de um website, visa fornecer um painel de controlo onde os utilizadores podem aderir a um laboratório. 
Os utilizadores podem controlar o hardware remoto. Será implementado um sistema hierárquico para 
fornecer diferentes funções, cada uma com as suas próprias permissões relativamente à forma como os utilizadores 
podem navegar pela informação fornecida pela aplicação web.

Este projeto implementará a infraestrutura de suporte à configuração, manipulação e visualização de hardware remoto. 
Baseado numa arquitetura com back-end (base de dados e Web \acs{api}) e front-end (Web App, com um dashboard).

%{\bf Palavras-chave:} lista de palavras-chave separadas por ;.

%% Página de agradecimentos
%\cleardoublepage
%\chapter*{Agradecimentos}
%Texto dos agradecimentos. É opcional.\\

% Geração do índice de conteúdos
\tableofcontents 

% Geração do índice de figuras
\cleardoublepage
\phantomsection
\addcontentsline{toc}{chapter}{List of Figures}
\listoffigures

% Geração do índice de listas
\cleardoublepage
\phantomsection
\addcontentsline{toc}{chapter}{List of Listings}
\lstlistoflistings

% Geração do índice de acronyms
\cleardoublepage
\phantomsection
\addcontentsline{toc}{chapter}{Acronyms}
\printacronyms

% Reiniciar a numeração de páginas
\setcounter{page}{1}
\pagenumbering{arabic}

% Capitulo 1
%
% Chapter 1
%
\chapter{Introduction}\label{cap:intro}
In recent years, the need for remote access to laboratory resources has grown significantly, driven by the expansion of online education, increased research collaboration, and the growing complexity of experimental setups. Traditional laboratories often require physical presence, which can limit accessibility and flexibility for students, researchers, and professionals. This limitation has become particularly evident in situations where geographical constraints, time restrictions, or extraordinary circumstances (such as global pandemics) prevent direct access to laboratory facilities.

The project described in this report addresses these challenges by developing a comprehensive solution for remote laboratory access that maintains the quality and integrity of hands-on experimentation while providing the flexibility of remote operation.

%
% Section 1.1
%
\section{Objectives}\label{sec:objectives}
Considering these emerging needs, a platform was proposed and implemented to enable secure, efficient, and user-friendly remote access to laboratory equipment and resources. The design of the platform was divided into two distinct phases. In the first phase, a database, an \acf{api}, and a web application were designed and implemented. This phase also encompassed the deployment architecture of the platform, including containerization, orchestration, and other essential configurations. In the second phase, the communication protocols between the platform and laboratory hardware were designed and implemented.

To design and implement a scalable platform for remote laboratory access, the following main objectives were established:
\begin{itemize}
\item Web \acs{api} to ensure comprehensive user, laboratory, and hardware management;
\item Web application that provides an intuitive and user-friendly interface;
\item Secure authentication and authorization mechanisms;
\item Role-based access control;
\item Robust data persistence through a well-designed database system;
\item Remote manipulation capabilities via terminal access to laboratory devices.
\end{itemize}

Additionally, several optional objectives were identified to enhance the system's functionality:
\begin{itemize}
\item Laboratory scheduling and reservation system;
\item Real-time visual monitoring of laboratory hardware;
\end{itemize}

\section{State of the Art}\label{sec:state_of_art}
In recent years, numerous initiatives have emerged to provide remote access to laboratory resources, particularly in higher education and research contexts. Pioneering projects such as \acs{mit}'s iLab~\cite{ilab} and LabShare~\cite{labshare} have demonstrated both the feasibility and substantial benefits of remote laboratories, enabling students and researchers to conduct experiments from anywhere in the world. These platforms typically emphasize secure access protocols, intelligent scheduling systems, and seamless integration with diverse laboratory equipment.

The existing literature underscores the critical importance of usability, scalability, and security in the design of remote laboratory systems. Key challenges identified include ensuring real-time interaction capabilities, maintaining hardware integration reliability, and providing adequate user support and training.

Several systems currently offer functionalities similar to those proposed in the Remote Lab project. The \acs{isa}, originally developed by \acs{mit}, represented an early successful implementation but is no longer operational and unavailable for public access~\cite{ilab}. Similarly, LabShare, another notable platform, is currently inactive~\cite{labshare}. Contemporary systems such as WebLab-Deusto focus on specialized domains like electronics and instrumentation, providing tailored interfaces and tools for remote experimentation in specific fields~\cite{weblabdeusto}.

Other significant contributions to the field include the Remote Laboratory Management System (RLMS)~\cite{rlms} and the Labshare Sahara framework~\cite{sahara}, which have influenced modern approaches to remote laboratory architecture and user experience design. These systems serve as valuable references for the development of the Remote Lab platform, informing critical decisions related to system architecture, user experience optimization, and hardware integration strategies.

%
% Section 1.3
%
\section{Document Structure}\label{sec:document_structure}
This report is organized as follows: Chapter~\ref{cap:proposed-architecture} presents and describes the proposed system architecture, including detailed specifications and core functionalities. Building upon the understanding of system components, the database design, web \acs{api} implementation, and web application development are described in Chapters~\ref{cap:database}, \ref{cap:web_api}, and \ref{cap:web_app}, respectively. Chapter~\ref{cap:project_org_deployment} details the project organization methodology and deployment strategies employed. Chapter~\ref{cap:experimental_results} presents comprehensive testing procedures and validation results for the implemented system. Finally, Chapter~\ref{cap:conclusions} provides conclusions regarding the system's performance and outlines potential directions for future development and enhancement. 

% Capitulo 2
%
% Chapter 3
%
\chapter{Proposed Architecture}\label{cap:proposed-architecture}

With the objective of implementing a platform to provide remote laboratory access, Figure~\ref{fig:architecture} presents a high-level architecture of the proposed system. A user can remotely access the platform through a web interface. The server hosting the platform communicates with an external authentication service to verify user credentials and establish secure sessions. Once authenticated, users can remotely manipulate laboratory hardware through the platform's interface.

\vspace{5mm}

\begin{figure}[H]
    \centering
    \includesvg[inkscapelatex=false, width=0.9\linewidth]{./../img/SimpleArchitectureRL.drawio.svg}
\caption{High-level Architecture}
\label{fig:architecture}
\end{figure}

This chapter is structured to provide a comprehensive understanding of the system's design. Section~\ref{sec:user_role} describes the different types of users and their respective interactions with the system. Section~\ref{sec:system_functionalities} introduces the core system functionalities and their implementation approach. Finally, Section~\ref{sec:system_components} presents the detailed system architecture with an introduction to its individual components and the technologies employed.

\section{System Users}\label{sec:user_role}
To ensure proper access control and functionality separation, the system defines three distinct user roles, each with specific permissions and capabilities:

\begin{description}
    \item[\textbf{Student}] can access assigned laboratories and manipulate their associated hardware within defined parameters and time constraints.
    \item[\textbf{Professor}] can create and configure laboratories, assign hardware resources, create and manage student groups, and monitor laboratory usage and performance.
    \item[\textbf{Administrator}] can manage system users, configure system-wide settings, monitor platform performance, and handle administrative aspects such as user provisioning and system maintenance.
\end{description}

This hierarchical separation of user types provides reliable access control throughout the system, determining whether a user can perform specific actions based on their assigned role. This approach is further enhanced through the implementation of a hierarchy and a \ac{rbac} approach, which provides fine-grained control over user permissions and system resources.

\section{System Functionalities}\label{sec:system_functionalities}
The system is built around several core functionalities that work together to provide a comprehensive remote laboratory experience. The platform supports robust user authentication and authorization mechanisms, ensuring secure access to laboratory resources. These security features are integrated throughout a Web \acs{api}, which exposes well-defined endpoints that provide access to resources, implement resource management capabilities, and enforce business logic rules.

The Web \acs{api} follows \ac{rest} principles and exposes resources through standardized \acp{uri}. It maintains bidirectional communication with a purpose-built database designed to meet the system's requirements for storing and managing all necessary information, including user data, laboratory configurations, and session logs.

A critical component of the system is the hardware abstraction layer, which communicates with the Web \ac{api} to translate high-level user instructions into hardware-specific commands. This abstraction enables the platform to support diverse types of laboratory equipment while maintaining a consistent user interface.

The Web Application serves as the primary user interface, making \ac{http} requests to the Web \ac{api} to provide users with comprehensive functionality, including:

\begin{itemize}
    \item Secure user authentication and profile management;
    \item Laboratory discovery, access, and real-time interaction capabilities;
    \item Group creation and collaborative workspace management;
    \item Hardware configuration and monitoring interfaces;
    \item Comprehensive user and system administration tools;
\end{itemize}

\section{System Components}\label{sec:system_components}
The proposed system comprises four main components: a Web Application, Web \acs{api}, Database, and Hardware Abstraction layer. Figure~\ref{fig:detailed_architecture} illustrates the detailed architecture of the proposed system along with the specific technologies considered during the design phase:

\begin{figure}[H]
    \centering
    \includesvg[inkscapelatex=false, width=1\linewidth]{./../img/SystemArchitectureWithTechRL.drawio.svg}
\caption{Detailed System Architecture}
\label{fig:detailed_architecture}
\end{figure}

The server entity shown in Figure~\ref{fig:detailed_architecture} is expanded in Figure~\ref{fig:detailed_architecture} to reveal its internal components: the Back-End (1), Database (2), and Front-End (3). The architecture also illustrates the external Authentication Server (4) and the Embedded Client (5), demonstrating the system's integration capabilities.

\subsection*{Back-End Architecture}
The Back-End (1) encompasses both the Web \acs{api} and the Hardware Abstraction layer. The Web \acs{api} utilizes \ac{http} as its primary communication protocol and is built using Spring Framework~\cite{spring-framework} as the main technology stack, with Kotlin~\cite{kotlin} as the primary programming language. Chapter~\ref{cap:web_api} provides comprehensive details about the Web \acs{api}, including the rationale for choosing Spring~\cite{spring} and Kotlin~\cite{kotlin}, architectural decisions, and implementation specifics.

\subsection*{Database Design}
The Database (2) is designed to efficiently store and manage all system information, including user profiles, laboratory configurations, hardware specifications, and operational logs. PostgreSQL~\cite{postgresql} was selected as the database management system. Chapter~\ref{cap:database} describes the database entity-relationship model, explains the rationale behind choosing PostgreSQL~\cite{postgresql}, and provides detailed implementation information.

\subsection*{Front-End Implementation}
The Front-End (3) is developed using Next.js~\cite{nextjs-docs}, a modern React-based framework~\cite{react} that provides server-side rendering capabilities and excellent performance optimization features. It serves as the primary visual interface for users and handles all client-server communication by making \acs{http} requests to the Back-End (1) for data retrieval and manipulation. Chapter~\ref{cap:web_app} provides an in-depth analysis of the Web Application design and implementation.

\subsection*{Authentication System}
User authentication is implemented through the Web Application using NextAuth.js~\cite{nextjs-authentication}, a comprehensive authentication framework that facilitates secure communication with external authentication providers. The Authentication Server (4) is provided by Microsoft \ac{oauth}~\cite{microsoft-oauth}, enabling users to authenticate using their institutional credentials.

\subsection*{Hardware Integration}
The Embedded Client (5) is illustrated to demonstrate the system's hardware communication capabilities. In the current implementation, the Embedded Client communicates with \ac{fpga} (COLOCAR REF), showcasing the platform's ability to interface with programmable hardware devices. The hardware abstraction layer within the Back-End (1) provides different instruction sets and interaction protocols depending on the specific hardware associated with each laboratory, ensuring compatibility with diverse equipment types while maintaining a consistent user experience.

% Capitulo 3
\chapter{Database} \label{cap:database}
The database serves as the foundational component of the system architecture. PostgreSQL was selected as the database management system due to its open-source nature and robust support for relational data models. This choice aligns with previous project implementations and provides the consistency and performance required for the system's operational needs.

This chapter presents an overview of the \ac{er model} and critical implementation details. Complete technical documentation is provided in the accompanying appendix.

\begin{figure}[H]
    \centering
    \includesvg[inkscapelatex=false, width=1\linewidth]{../img/DB-diagrams/ERDiagramRL.drawio.svg}
    \caption{Entity-Relationship Model (ER Model)}
    \label{fig:er_model}
\end{figure}

The database design follows a normalized relational structure that supports user authentication, secure session management, and the remaining system functionalities. The ER model encompasses the core entities required for system functionality while maintaining data integrity and scalability.

\section{Core Entities}
\subsection*{User}

\begin{figure}[H]
    \centering
    \includesvg[inkscapelatex=false, width=0.5\linewidth]{../img/DB-diagrams/User.drawio.svg}
    \caption{User Entity}
    \label{fig:user_entity}
\end{figure}

The \textbf{User} entity represents a user in the system. The username and email attributes are provided by the authentication system. The role serves as descriminator attribute to identify whether the user is an administrator, professor or student.

\subsection*{Token}

\begin{figure}[H]
    \centering
    \includesvg[inkscapelatex=false, width=0.6\linewidth]{../img/DB-diagrams/Token.drawio.svg}
    \caption{Token Entity}
    \label{fig:token_entity}
\end{figure}

A user can create \textit{N} tokens. The \textbf{Token} is a weak entity because it cannot be uniquely identified by its attributes alone and therefore requires a user, which is a strong entity, to be identified. A token is created by only one user.

This is a useful entity for authentication propuses. It was designed to hold a hash value in the \textit{token\_validation} attribute.

Authentication workflow:

\begin{enumerate}
    \item Upon successful user login, a unique token is created with cryptographically secure values and stored in the database.
    \item For subsequent authenticated operations, the system queries the database to verify the client-provided token against stored values.
    \item Valid tokens enable secure user identification without transmitting unique identifiers.
\end{enumerate}

The \textit{last\_used\_at} and the \textit{created\_at} are useful for determining token expiration.

\subsection*{Laboratory}

\begin{figure}[H]
    \centering
    \includesvg[inkscapelatex=false, width=0.55\linewidth]{../img/DB-diagrams/Laboratory.drawio.svg}
    \caption{Laboratory Entity}
    \label{fig:laboratory_entity}
\end{figure}

A user, as an administrator or professor, can create \textit{N} laboratories. When creating a \textbf{Laboratory}, the user can define the name (\textit{lab\_name}) and description (\textit{lab\_description}). They can also define the duration of a laboratory session (\textit{lab\_duration}) and its queue limit (\textit{lab\_queue\_limit}).

\subsection*{Hardware}

\begin{figure}[H]
    \centering
    \includesvg[inkscapelatex=false, width=0.55\linewidth]{../img/DB-diagrams/Hardware.drawio.svg}
    \caption{Hardware Entity}
    \label{fig:hardware_entity}
\end{figure}

Upon successful laboratory creation, the user can associate \textbf{Hardware} to it, which must be created separately.

For the creation, it requires a name (\textit{hw\_name}), IP (\textit{ip\_address}) and MAC (\textit{mac\_address}) addresses (which can be null depending on the hardware), a status (\textit{hw\_status}) to indicate whether the hardware is under maintenance, occupied, or available, and a serial number (\textit{hw\_serial\_num}) to uniquely identify the hardware. Although it has an ID, the serial number helps physically identify the hardware.    

\subsection*{Group}

\begin{figure}[H]
    \centering
    \includesvg[inkscapelatex=false, width=0.5\linewidth]{../img/DB-diagrams/Group.drawio.svg}
    \caption{Group Entity}
    \label{fig:group_entity}
\end{figure}

For a student to access a laboratory, they must be in a group that is associated with that laboratory. A professor can create a \textbf{Group} and associate users to it.

When creating a group, the user needs to name it (\textit{group\_name}) and, optionally, add a description (\textit{group\_description}) to it.

\subsection*{Lab Session}

\begin{figure}[H]
    \centering
    \includesvg[inkscapelatex=false, width=0.45\linewidth]{../img/DB-diagrams/LabSession.drawio.svg}
    \caption{Lab Session Entity}
    \label{fig:lab_session_entity}
\end{figure}

Finally, a user can join a laboratory if they are in a group associated with it. If the laboratory is being used, the user enters a waiting queue; otherwise, a \textbf{Lab Session} is created.

Lab Session is a weak entity. It requires two strong entities to be identified: the \textbf{User} entity and the \textbf{Laboratory} entity. This is used to check whether a user is in a lab session or for statistical purposes. The \textit{state} attribute indicates whether the session is over or still running. The \textit{start\_time} and \textit{end\_time} can be used for statistical details, such as determining how much time a user spent in a laboratory, or for future purposes, such as scheduling sessions.

\section{Implementation Details}
Although PostgreSQL is being used for its functionalities, it was decided that all logic and verifications are implemented in the Web API, so that no triggers or complex constraints are implemented on the database side.

\section{Summary}
This chapter has provided an overview of the database architecture, implementation, and design decisions. It has also presented the \acs{er model} of the database and described a typical user journey, explaining database interactions.

The documentation should be consulted for a comprehensive deep dive. It explains every entity, its attributes, and provides theoretical insights.

% Capitulo 4
\chapter{Web API} \label{cap:web_api}
The Web API provides endpoints for user management, authentication, authorization, and CRUD operations. 

The API is developed with Kotlin and Spring Boot, and follows the Controller-Service-Repository pattern, which is prevalent in many Spring Boot applications. We chose this pattern because of the separation of concerns it provides and the possibilities for unit testing.

To make the codebase even easier to maintain and improve the quality of life during development, Spring Framework's Inversion of Control container (COLOCAR REFERENCIA) and the Strategy pattern principle were also used. 

Spring's dependency injection is a well-known technology in Java enterprise programming. It provides an easy way to declare dependencies, since the API was mostly built following Object Oriented Programming (OOP) principles. This framework allows us to declare the necessary dependencies for each module. It also provides a BeanFactory interface for advanced configurations. Using these Spring technologies moves the object management to Spring.

The Strategy pattern allowed us to have more control over the specific implementation since it follows an interface. Every concrete implementation follows an interface, making it possible to change a class dynamically without changing the code. Spring's dependency injection works very well with this strategy design pattern. This makes unit tests much easier when the concrete implementation is not intended to be tested without changing it's code.

\begin{figure}[H]
    \centering
    \includesvg[inkscapelatex=false, width=0.9\linewidth]{../img/api/api_architecture.drawio.svg}
    \caption{API Architecture}
    \label{fig:api_architecture}
\end{figure}

Figure \ref{fig:api_architecture} provides a simple overview of the implemented \acs{api}. The \textbf{HTTP} module (Controller) is responsible for exposing the endpoints and handling the messages. When a request is made, the HTTP module receives the request and hands it to the \textbf{Service} module. This is where the logic and verifications are performed. Since it is necessary to fetch and save data, a \textbf{Repository} module is needed. The repository module is responsible for communicating with the database.

\vspace{3mm}

\begin{figure}[H]
    \centering
    \includesvg[inkscapelatex=false, width=0.9\linewidth]{../img/api/detailed_architecture.drawio.svg}
    \caption{API Detailed Architecture}
    \label{fig:api_detailed_architecture}
\end{figure}

Figure \ref{fig:api_detailed_architecture} provides a more detailed overview of how the architecture is composed. 

As explained, the HTTP module contains the controllers, each one with its functions. The pipeline contains the argument resolvers and interceptors. For the implemented system, only one argument resolver and two interceptors were implemented. The argument resolver (COLOCAR REFERENCIA) is used to provide user information to the controllers. Since the authentication method we used was token-based, this argument resolver extracts user information from the request. Every controller that has a parameter with the type \textit{AuthenticatedUser} will be authenticated. 

For the request to contain the needed information about the user, an interceptor is required. This is one of the two interceptors implemented. Every request, before reaching the controller, passes through every configured interceptor. This authentication interceptor checks if the handler parameters contain a parameter of the type \textit{AuthenticatedUser}. If it does, the entire process of getting the token from the request, verifying it, and retrieving the user is performed. If not, normal execution continues.
\begin{center}
    \begin{lstlisting}[caption={Type AuthenticatedUser verification example}]
    if (handler is HandlerMethod &&
        handler.methodParameters.any {
            it.parameterType == AuthenticatedUser::class.java
        }
    )
    \end{lstlisting}
\end{center}

The other interceptor is for an API key. It checks if the handler contains a custom annotation. If yes, the API key is validated; if not, an unauthorized response is sent. This interceptor is useful for the login endpoint. This login endpoint is to be performed in the Web App and is not meant to be used by end users.

The service module performs the necessary checks, using domain classes defined in the Domain module. These classes provide configurations and methods for validating certain data. Configurations in domain classes are provided by a JSON file containing domain restrictions.

\begin{center}
    \begin{lstlisting}[caption={Example of the group entry}]
        "group": {
            "groupName": {
              "min": 3,
              "max": 100,
              "optional": false
            },
            "groupDescription": {
              "min": 10,
              "max": 1000,
              "optional": true
            }
          }
    \end{lstlisting}
\end{center}

This JSON file is converted to a class using Kotlin Serialization. This allows an easy way to change specific values without touching the codebase.

\vspace{3mm}

Every response, whether successful or an error, follows a specific format. The API documentation provides an overview of the possible responses. Error messages follow the application/problem+json format (COLOCAR REFERENCIA).

\vspace{3mm}

The API is expected to be public, providing full documentation (COLOCAR REFERENCIA POSTMAN) in Postman. It was decided to have the documentation in Postman because of the easy-to-use documentation builder inside the collection containing the tests for the endpoints. The API key is implemented for this reason. In future work, when the API reaches a stable version to be made public, users who want to use it will need to log in to the website and generate a token to use the API.

% Capitulo 5
\include{5-web-app/5-web-app}

% Capitulo 6
\chapter{Project Organization and Deployment} \label{cap:project_org_deployment}

% Capitulo 7
\chapter{Experimental Results} \label{cap:experimental_results}

% Capitulo 8
\chapter{Conclusions}
\label{cap:conclusions}

This report has presented the design, implementation, and validation of a comprehensive platform for remote laboratory access, addressing the growing need for flexible and accessible laboratory resources in educational and research environments. The solution successfully bridges the gap between traditional hands-on experimentation and the requirements of modern remote learning and collaboration.

The developed platform successfully fulfills all primary objectives established at the project's inception. The implementation of a robust web \acs{api} provides comprehensive management capabilities for users, laboratories, and hardware resources. The intuitive web application interface ensures accessibility for users with varying technical backgrounds, while the secure authentication and authorization mechanisms, coupled with role-based access control, guarantee system integrity and appropriate resource allocation.

The well-designed database system ensures reliable data persistence, supporting the platform's scalability requirements. Furthermore, the implementation of remote manipulation capabilities of laboratory devices enables real-time interaction with experimental equipment, maintaining the essential hands-on experience that characterizes quality laboratory work.

\section{Limitations and Future Work}
\label{sec:limitations_future_work}

While the current implementation successfully addresses the primary objectives, several areas present opportunities for future enhancement. The expansion of the testing framework to include comprehensive load testing will provide valuable insights into system performance under high-usage conditions. This analysis will inform optimization strategies and guide infrastructure scaling decisions.

Future development efforts should focus on expanding hardware compatibility to support a broader range of laboratory equipment types. Additionally, the implementation of advanced monitoring and analytics capabilities could provide valuable usage insights and facilitate resource optimization. The integration of collaborative features, such as shared experimentation sessions and real-time collaboration tools, would further enhance the platform's educational value.

Furthermore, the development of mobile applications and the enhancement of accessibility features would broaden the platform's reach and usability across diverse user populations.


%%
% Chapter 4
%
\chapter{Implemented Infrastructure} \label{cap:implemented-infrastructure}

This chapter details the infrastructure implemented for the Remote Lab platform, covering the main components, technologies, deployment strategy, and integration between system modules.

\section{Overview}

The Remote Lab platform is designed as a modular, containerized system that enables secure and efficient remote access to laboratory equipment. The infrastructure follows a layered architecture, separating the user interface, backend logic, and hardware integration, and is built with scalability and maintainability in mind.

\begin{figure}[H]
    \centering
    \includesvg[inkscapelatex=false, width=0.95\linewidth]{../img/SystemArchitectureWithTechRL.drawio.svg}
    \caption{System Architecture Overview}
    \label{fig:system-architecture}
\end{figure}

\section{Project Structure}

The Remote Lab project is organized into several main directories, each of which is managed as a GitHub submodule. This approach allows for independent development, versioning, and access control of each core component, supporting both modularity and security. The use of submodules also facilitates collaboration among different teams and ensures that sensitive information is handled appropriately.

The main submodules of the project are:

\begin{itemize}
    \item \textbf{api/} -- Contains the backend source code, implemented in Kotlin with Spring Boot. This submodule is responsible for business logic, user management and laboratory session control.
    \item \textbf{db/} -- Includes database scripts, migrations, and documentation, supporting the persistence layer of the system.
    \item \textbf{docs/} -- Stores project documentation, including technical reports, user guides, and architectural diagrams.
    \item \textbf{img/} -- Stores project images, including technical reports, user guides, and architectural diagrams.
    \item \textbf{nginx/} -- This directory is not a GitHub submodule but it provides Nginx configuration files for reverse proxying, load balancing, and secure access to backend services.
    \item \textbf{private/} -- Dedicated to sensitive files and configurations, such as environment variables and secrets necessary for the secure operation of the system, and specific to our implementation choices. This submodule contains the information and configuration files required to run the project with our selected authentication (login) and database setup, reflecting the particular options chosen for our use case. It is not included directly in the main repository, ensuring that only authorized members have access to confidential information like API keys, external service credentials, and other private data essential for both production and development environments.
    \item \textbf{website/} -- Holds the frontend web application, built with Next.js (React). This submodule provides the user interface for laboratory access, scheduling, and management.
    \item \textbf{wiki/} -- Stores the GitHub Wiki pages, including the project documentation, deployment instructions, and other relevant information.
\end{itemize}

This modular structure, based on GitHub submodules, allows for independent development, testing, and deployment of each component, supporting both scalability and maintainability. By clearly separating concerns and leveraging best practices such as containerization and secure secret management, the project is well-positioned for collaborative development and future expansion. 

\section{Implementation Details}

Key implementation decisions and details include:

\begin{itemize}
    \item \textbf{Containerization:} All major components (frontend, backend, database) are containerized using Docker, ensuring consistent environments across development and production.
    \item \textbf{Orchestration:} Docker Compose is used to manage multi-container deployments, networking, and environment configuration.
    \item \textbf{Backend:} The backend uses JDBI for type-safe database access and is configured via environment variables for flexibility and security.
    \item \textbf{Frontend:} The frontend is built with Next.js, providing a modern, responsive interface and integrating with the backend via RESTful APIs.
    \item \textbf{Authentication:} Microsoft OAuth (via NextAuth) is used for secure authentication, enabling university login. The system is designed to be flexible, so other OAuth providers or even an internal login mechanism could be used if required.
    \item \textbf{Role-based Access Control:} The system enforces permissions based on user roles, ensuring secure and appropriate access to resources.
    \item \textbf{Hardware Abstraction:} The backend abstracts hardware-specific details, allowing for easy extension to new laboratory equipment.
    \item \textbf{Automation:} The \texttt{start.sh} script automates the bootstrap process, starting all necessary services with a single command.
\end{itemize}

\subsection*{Role-Based Access Control (RBAC)}

The system implements Role-Based Access Control (RBAC) to ensure that users have access only to the resources and actions appropriate for their role. Each authenticated user is assigned a role, such as student, professor, or administrator, which determines their permissions within the platform. 

Roles are enforced both on the backend and frontend. On the backend, endpoints and business logic check the user's role before allowing access to sensitive operations, such as managing laboratory sessions, accessing administrative features, or modifying user data. On the frontend, the user interface dynamically adapts to the user's role, displaying only the features and options relevant to their permissions.

The main roles in the system are:
\begin{itemize}
    \item \textbf{Student:} Can view and book laboratory sessions, access their own session history, and interact with laboratory equipment during their reserved times. Students have access only to features relevant to their participation in laboratory activities.
    \item \textbf{Professor:} In addition to all student permissions, professors can create and manage laboratory sessions, view and manage student participation, and access additional data and reports related to their classes or laboratories.
    \item \textbf{Administrator:} Has full access to all system features, including user management, system configuration, and oversight of all laboratory sessions and resources. Administrators can manage roles, permissions, and perform maintenance or troubleshooting tasks across the platform.
\end{itemize}

This approach provides a secure and flexible way to manage access, making it easy to introduce new roles or adjust permissions as the platform evolves. The RBAC system is central to maintaining the integrity and security of the Remote Lab environment.

These choices ensure the platform is robust, extensible, and easy to deploy or develop locally.

In addition, the web application allows users to view and interact with the platform as if they had a lower role than their own. This feature is particularly useful for testing, support, and understanding the user experience from different perspectives. For example, an administrator or professor can switch to a student view to verify permissions, troubleshoot issues, or provide guidance, without needing to log in as a different user. 

\section{Web Application}

The web application provides a modern, user-friendly interface that enables users to access, schedule, and manage laboratory resources remotely. Built with Next.js (React), the web app is designed to be responsive and accessible from any device, ensuring a seamless experience for students, professors, and administrators.

\subsection{Overview}
The web application serves as the primary point of interaction for users, integrating with the backend services via RESTful APIs. It supports multiple user roles, each with tailored access and features according to their permissions.

The choice of Next.js as the framework for the web application was motivated by several factors. Next.js is built on top of React, a technology already taught in the course, which allowed us to leverage our existing knowledge. At the same time, Next.js simplifies development by providing built-in solutions for routing, server-side rendering, API integration, and other common tasks. This not only accelerated our development process but also allowed us to work with a modern, widely adopted framework in the industry, expanding our understanding of full-stack web development and exposing us to new possibilities and best practices.

The web application also retrieves the domain configuration in JSON format from the API. By consuming this configuration, the frontend ensures that its domain-related settings are always synchronized and up to date with those defined on the backend. This approach centralizes domain management and reduces the risk of inconsistencies between the client and server.

\begin{figure}[H]
    \begin{center}
        \resizebox{15cm}{!}{\includesvg{../img/SimpleArchitectureRL.svg}}
    \end{center}
    \caption{Web Application High-Level Architecture}
    \label{fig:webapp_simple_architecture}
\end{figure}

Figure \ref{fig:webapp_simple_architecture} illustrates the high-level architecture of the web application, showing the main components and their interactions.

\subsection{Client-Server Logic Separation}

The web application leverages Next.js to achieve a clear separation between client-side and server-side logic. All requests to the backend API are made from the server side, ensuring that sensitive operations and data exchanges are handled securely and are not exposed directly to the client. This architecture enhances security, enables better control over data flow, and allows for efficient server-side rendering and data fetching. For more details on using Next.js to perform API requests from the server, see~\cite{auth0-nextjs-server-actions}.

\subsection{Main Features}
\begin{itemize}
    \item \textbf{Authentication:} Secure login using Microsoft OAuth (NextAuth), supporting university credentials and role assignment.
    \item \textbf{Dashboard:} Personalized dashboard displaying relevant information, upcoming sessions, and quick access to key features.
    \item \textbf{Laboratory Management:} Professors and administrators can create, edit, and manage laboratory sessions, equipment, and participant lists.
    \item \textbf{Calendar and Scheduling:} Interactive calendar for booking and managing laboratory sessions, with real-time availability and notifications.
    \item \textbf{Role-Based Access:} Interface adapts to the user's role, showing only the features and data relevant to their permissions. Users with higher roles can view and interact with the platform as if they had a lower role for testing and support purposes.
    \item \textbf{Responsive Design:} Optimized for desktops, tablets, and mobile devices, ensuring accessibility and usability across platforms.
\end{itemize}

\subsection{Integration and Security}
The web application communicates securely with the backend via RESTful APIs, using authentication tokens to protect sensitive operations. User sessions and data are managed according to best practices, ensuring privacy and integrity. The frontend is designed to prevent unauthorized access and to provide a robust, extensible foundation for future enhancements. 

\section{Deployment}

The deployment process for the Remote Lab platform is designed to be straightforward, secure, and reproducible, leveraging modern DevOps practices and containerization technologies.

\subsection*{Containerization and Orchestration}
All major components of the platform—including the backend (api), frontend (website), and database—are containerized using Docker. This ensures consistency across development, testing, and production environments. Docker Compose is used to orchestrate multi-container deployments, manage networking between services, and handle environment-specific configurations.

\subsection*{Environment Configuration and Secrets}
Sensitive configuration files and environment variables required for deployment are managed in the \textbf{private/} submodule. This submodule contains the necessary secrets, such as API keys, database credentials, and authentication settings, tailored to the specific requirements of the platform. Access to this submodule is restricted to authorized team members, ensuring the security of confidential information.

\subsection*{Automation with start.sh}

To further streamline the deployment process, the platform provides a \texttt{start.sh} script located at the root of the repository. This script automates the bootstrap process by orchestrating the initialization of all required services and dependencies with a single command. It handles tasks such as building Docker images, starting containers using Docker Compose, and ensuring that environment variables and configuration files are correctly loaded from the \texttt{private/} submodule.

The \texttt{start.sh} script also supports several flags to customize the deployment process, such as selecting the environment (development or production), starting only the API, enabling Cloudflare tunneling, or switching branches. These options make it easy to adapt the deployment to different scenarios with simple command-line arguments.

\subsection*{Deployment Steps}
\begin{enumerate}
    \item \textbf{Clone the Repository and Submodules:} Clone the main repository and initialize all submodules, including \texttt{private/}, to ensure all components and configurations are available.
    \item \textbf{Configure Environment Variables:} Ensure that all required environment variables and secret files are present in the appropriate locations, as provided by the \texttt{private/} submodule.
    \item \textbf{Build and Start Services:} Use the provided \texttt{docker-compose.yml} file to build and start all services with a single command (e.g., \texttt{docker compose up --build}).
    \item \textbf{Access the Platform:} Once all containers are running, the platform can be accessed via the configured web address. Nginx is used as a reverse proxy to route traffic securely to the appropriate services.
\end{enumerate}

\subsection*{Local and Production Deployment}
The deployment process is designed to be nearly identical for both local development and production environments. Developers can run the entire stack locally using Docker Compose, mirroring the production setup. For production, additional considerations such as SSL certificates, domain configuration, and scaling may be applied, but the core process remains the same.

This approach ensures that deployments are reliable, repeatable, and secure, minimizing the risk of configuration drift and simplifying both initial setup and ongoing maintenance.


\section{Technologies Used}

\begin{itemize}
    \item \textbf{Frontend:} Implemented with Next.js (React framework), providing a modern, responsive web interface for users to interact with laboratories, schedule sessions, and control hardware.
    \item \textbf{Backend:} Developed in Kotlin using Spring Boot, exposing RESTful APIs for user management, authentication, laboratory session control, and business logic enforcement.
    \item \textbf{Database:} PostgreSQL is used to persist user data, session information, access logs, and configuration settings.
    \item \textbf{ORM/Database Access:} JDBI is used for type-safe, modular database access in the backend.
    \item \textbf{Authentication:} Microsoft OAuth via NextAuth is used for user authentication, supporting multiple roles (student, professor, administrator).
    \item \textbf{Containerization:} Docker is used to containerize all major components (frontend, backend, database), ensuring consistent deployment across environments.
    \item \textbf{Orchestration:} Docker Compose manages multi-container deployment, networking, and environment configuration.
\end{itemize}

\section{System Components}

\begin{itemize}
    \item \textbf{Web Application (Frontend):} Provides dashboards, laboratory access, real-time hardware monitoring, and session management. Built with Next.js and deployed as a Docker container.
    \item \textbf{API Server (Backend):} Handles authentication, authorization, laboratory and user management, and hardware abstraction. Built with Kotlin and Spring Boot, also containerized.
    \item \textbf{Database:} PostgreSQL instance running in a Docker container, with persistent storage volumes.
    \item \textbf{Hardware Abstraction Layer:} Backend modules abstract hardware-specific details, exposing unified interfaces for laboratory equipment control.
\end{itemize}

\section{Deployment Architecture}

The system is deployed using Docker Compose, which defines and manages the following services:

\begin{itemize}
    \item \textbf{db:} PostgreSQL database container, with health checks and persistent volumes.
    \item \textbf{api:} Backend API container, built from the Kotlin/Spring Boot project, depending on the database service.
    \item \textbf{website:} Frontend container, built from the Next.js project, depending on the API service.
\end{itemize}

All services are connected via Docker networks to ensure secure and efficient communication. Environment variables and secrets are managed via \texttt{.env} files.

\section{Build and CI/CD}

\begin{itemize}
    \item \textbf{Gradle:} Used for building and managing backend dependencies.
    \item \textbf{NPM:} Used for frontend dependency management and builds.
    \item \textbf{Dockerfiles:} Multi-stage builds are used for both backend and frontend to optimize image size and security.
    \item \textbf{GitHub Actions:} (If applicable) Used for continuous integration and automated builds.
\end{itemize}

\section{Notable Implementation Details}

\begin{itemize}
    \item The backend uses JDBI for database access, configured with application-specific requirements.
    \item Environment variables are used to configure database connections and secrets, improving security and flexibility.
    \item The system supports role-based access control, with different permissions for students, professors, and administrators.
    \item The hardware abstraction layer allows for future extension to new types of laboratory equipment.
\end{itemize}

\section{Summary}

The implemented infrastructure leverages modern web technologies, containerization, and modular design to provide a robust, scalable, and maintainable platform for remote laboratory access.


% Referências
\bibliographystyle{unsrt}
\bibliography{references}
\addcontentsline{toc}{chapter}{References}

% Apêndices (opcional)
%\appendix
%\include{apendiceex}

\end{document}