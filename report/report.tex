% Classe do documento e parâmetros gerais.
\documentclass[a4paper,openright,twoside,11pt]{report}

% Packages utilizadas e respetivos parâmetros.
\usepackage[utf8]{inputenc}
\usepackage[english]{babel}
\addto{\captionsenglish}{\renewcommand{\bibname}{References}}
\addto{\captionsenglish}{\renewcommand{\contentsname}{Contents}}
\addto{\captionsenglish}{\renewcommand{\appendixname}{Appendix}}

\usepackage{lipsum} % gerador de texto
\usepackage{graphicx}
\usepackage{url}
\usepackage[Algoritmo]{algorithm}
\usepackage{algorithmicx}
\usepackage{algpseudocode}
\renewcommand{\algorithmicrequire}{\textbf{Dados: }}
\renewcommand{\algorithmicensure}{\textbf{Resultado: }}

% Definições das dimensões das páginas
\setlength{\textheight}{24.00cm}
\setlength{\textwidth}{15.50cm}
\setlength{\topmargin}{0.35cm}
\setlength{\headheight}{0cm}
\setlength{\headsep}{0cm}
\setlength{\oddsidemargin}{0.25cm}
\setlength{\evensidemargin}{0.25cm}

%\renewcommand{\baselinestretch}{1}

% Página inicial (capa)
\title{
   \vspace{-50mm}
   \begin{minipage}[l]{\textwidth}
      \hspace{-20mm}\resizebox{75mm}{!}{\includegraphics{../../img/figures/logoISELnew2.png}}\\
   \end{minipage}\\[20mm]
   {\bf Remote Lab}
}

% Nome dos autores (um por linha)
\author{
\begin{tabular}{ll}
             & António Alves  \\
             & Ângelo Azevedo \\[50mm]
\end{tabular}}

\date{
\begin{tabular}{ll}
  {Advisor:} & Pedro Matutino \\
\end{tabular}\\[10mm]
% Deixar o indicador respetivo em função da versão do relatório.
Report for Project and Seminar Class\\
Computer Science and Computer Engineering BSc\\[20mm]
June 2025}

\begin{document}
\pagenumbering{roman}
\thispagestyle{empty}
\maketitle

\baselineskip 18pt % line spacing: 12pt for single, 18pt for 1 1/2, and 24pt for double spacing

\newpage
\thispagestyle{empty}
% Fim da contracapa

% Página com identificação completa (número e nome) e assinaturas do(s) estudante(s) e do(s) orientador(es)
\cleardoublepage
\setcounter{page}{1}
\begin{center}
\textsc{\LARGE Lisbon School of Engineering}\\[50mm]

{\large \bf  Remote Lab}\\[20mm]

\begin{tabular}{rl}
  50539  & António Alves\\[10mm]
           & \rule{75mm}{0.5pt}\\[5mm]
  50565  & Ângelo Azevedo\\[10mm]
           & \rule{75mm}{0.5pt}\\
\end{tabular}\\[10mm]

\begin{tabular}{rl}
  Advisor: & Pedro Matutino\\[10mm]
                & \rule{75mm}{0.5pt}\\[5mm]
\end{tabular}\\[10mm]

Report for Project and Seminar Class\\
Computer Science and Computer Engineering BSc\\[20mm]
June 2025\\
\end{center}

% Página de resumo em Português
\cleardoublepage
\chapter*{Abstract}
The design, development, implementation, and finally, the validation of digital systems re-
quire, in addition to simulators, the use of hardware to verify their implementations in real
devices. In the current teaching paradigm, in which face-to-face time is reduced and remote
and autonomous work is increased, it is necessary to create alternatives to the current model.
The Remote Lab project aims to provide a virtual lab with access to remote hardware.
This lab consists of a web application running on an embedded system. The web application,
accessed via a website, aims to provide a dashboard where users can join a laboratory. This
is where users can control the remote hardware. A hierarchy system will be implemented to
provide different roles, each with their own permissions relative to how users can browse the
information provided by the web application.

%{\bf Palavras-chave:} lista de palavras-chave separadas por ;.

%% Página de agradecimentos
%\cleardoublepage
%\chapter*{Agradecimentos}
%Texto dos agradecimentos. É opcional.\\

% Geração do índice de conteúdos
\cleardoublepage
\tableofcontents \cleardoublepage

% Geração do índice de figuras e de tabelas
\listoffigures \cleardoublepage
\listoftables \cleardoublepage

% Reiniciar a numeração de páginas
\setcounter{page}{1}
\pagenumbering{arabic}

% Capitulo 1
%
% Chapter 1
%
\chapter{Introduction} \label{cap:intro}

%
% Secção 1.1
%
\section{Context and Motivation} \label{sec11}
In recent years, the need for remote access to laboratory resources has grown significantly, driven by the expansion of online education, research collaboration, and the increasing complexity of experimental setups. Traditional laboratories often require physical presence, which can limit accessibility and flexibility for students, researchers, and professionals. The \textbf{Remote Lab} project aims to address these challenges by providing a platform that enables secure, efficient, and user-friendly remote access to laboratory equipment and resources.

%
% Secção 1.2
%
\section{Objectives} \label{sec12}
The main objectives of the Remote Lab project are:
\begin{itemize}
    \item To design and implement a scalable platform for remote laboratory access.
    \item To ensure secure authentication and authorization for different user roles.
    \item To provide an intuitive user interface for managing and scheduling laboratory sessions.
    \item To support integration with various types of laboratory hardware.
\end{itemize}

%
% Secção 1.3
%
\section{Scope} \label{sec13}
This project focuses on the development of the core platform, including backend services, user management, and basic hardware integration. Advanced features such as real-time data analytics, support for a wide range of laboratory devices, and extensive reporting capabilities are considered out of scope for the current phase.

%
% Secção 1.4
%
\section{Methodology} \label{sec14}
The project follows a modular and iterative development approach, leveraging modern software engineering practices. The backend is implemented using Kotlin and follows a layered architecture, while the frontend is developed with Next.js to provide a responsive and accessible user experience.

%
% Secção 1.5
%
\section{Structure of the Document} \label{sec15}
The remainder of this report is organized as follows:
\begin{itemize}
    \item \textbf{Chapter 2:} Related Work – Overview of existing solutions and technologies.
    \item \textbf{Chapter 3:} System Architecture – Description of the overall system design.
    \item \textbf{Chapter 4:} Implementation – Details of the main components and their interactions.
    \item \textbf{Chapter 5:} Evaluation – Assessment of the system's performance and usability.
    \item \textbf{Chapter 6:} Conclusions and Future Work – Summary of achievements and directions for future development.
\end{itemize}

% Capitulo 2
%
% Chapter 2
%
\chapter{Placement} \label{cap:placement}

This chapter is organized into two sections, where we describe related work and some systems similar to the one developed in this project.

\section{Related Work}
In recent years, several initiatives have emerged to provide remote access to laboratory resources, especially in the context of higher education and research. Projects such as MIT's iLab and LabShare have demonstrated the feasibility and benefits of remote laboratories, enabling students and researchers to conduct experiments from anywhere in the world. These platforms typically focus on providing secure access, scheduling, and integration with a variety of laboratory equipment. The literature highlights the importance of usability, scalability, and security in the design of such systems, as well as the challenges associated with real-time interaction and hardware integration.

\section{Similar Systems}
There are several systems that offer functionalities similar to those of the Remote Lab project. For example, the iLab Shared Architecture (ISA) provides a framework for sharing laboratory equipment over the internet, supporting both batch and interactive experiments. LabShare is another notable example, offering a collaborative platform for remote experimentation and resource sharing among institutions. Other systems, such as WebLab-Deusto and VISIR, focus on specific domains like electronics and instrumentation, providing specialized interfaces and tools for remote experimentation. These systems serve as valuable references for the development of the Remote Lab platform, informing decisions related to architecture, user experience, and integration with laboratory hardware.


% Capitulo 3
\include{proposed-architecture/proposed-architecture}

% Capitulo 4
%
% Chapter 4
%
\chapter{Implemented Infrastructure} \label{cap:implemented-infrastructure}
%\include{capexemplos}

% Capitulo 4
%\include{captestes}

% Referências
\bibliographystyle{unsrt}
\bibliography{references}
\addcontentsline{toc}{chapter}{References}

% Apêndices (opcional)
\appendix
%\include{apendiceex}

\end{document}