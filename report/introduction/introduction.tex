%
% Chapter 1
%
\chapter{Introduction} \label{cap:intro}

%
% Secção 1.1
%
\section{Context and Motivation} \label{sec11}
In recent years, the need for remote access to laboratory resources has grown significantly, driven by the expansion of online education, research collaboration, and the increasing complexity of experimental setups. Traditional laboratories often require physical presence, which can limit accessibility and flexibility for students, researchers, and professionals. The \textbf{Remote Lab} project aims to address these challenges by providing a platform that enables secure, efficient, and user-friendly remote access to laboratory equipment and resources.

%
% Secção 1.2
%
\section{Objectives} \label{sec12}
The main objectives of the Remote Lab project are:
\begin{itemize}
    \item To design and implement a scalable platform for remote laboratory access.
    \item To ensure secure authentication and authorization for different user roles.
    \item To provide an intuitive user interface for managing and scheduling laboratory sessions.
    \item To support integration with various types of laboratory hardware.
\end{itemize}

%
% Secção 1.3
%
\section{Scope} \label{sec13}
This project focuses on the development of the core platform, including backend services, user management, and basic hardware integration. Advanced features such as real-time data analytics, support for a wide range of laboratory devices, and extensive reporting capabilities are considered out of scope for the current phase.

%
% Secção 1.4
%
\section{Methodology} \label{sec14}
The project follows a modular and iterative development approach, leveraging modern software engineering practices. The backend is implemented using Kotlin and follows a layered architecture, while the frontend is developed with Next.js to provide a responsive and accessible user experience.

%
% Secção 1.5
%
\section{Structure of the Document} \label{sec15}
The remainder of this report is organized as follows:
\begin{itemize}
    \item \textbf{Chapter 2:} Related Work – Overview of existing solutions and technologies.
    \item \textbf{Chapter 3:} System Architecture – Description of the overall system design.
    \item \textbf{Chapter 4:} Implementation – Details of the main components and their interactions.
    \item \textbf{Chapter 5:} Evaluation – Assessment of the system's performance and usability.
    \item \textbf{Chapter 6:} Conclusions and Future Work – Summary of achievements and directions for future development.
\end{itemize}