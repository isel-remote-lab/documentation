\documentclass[a4paper,twoside,11pt]{article}
\usepackage[utf8]{inputenc}
\usepackage[portuguese]{babel}
\usepackage{graphicx}
\usepackage{url}

% pdflatex

% redefinição das margens das páginas
\setlength{\textheight}{24.00cm}
\setlength{\textwidth}{15.50cm}
\setlength{\topmargin}{0.35cm}
\setlength{\headheight}{0cm}
\setlength{\headsep}{0cm}
\setlength{\oddsidemargin}{0.25cm}
\setlength{\evensidemargin}{0.25cm}

\title{Laboratório Remoto}

\author{
\begin{tabular}{c}
             Ângelo Azevedo, n.º 50565, e-mail: a50565@alunos.isel.pt, tel.: 967572352\\
             António Alves, n.º 00000, e-mail: a00000@alunos.isel.pt, tel.: 000000000\\
\end{tabular}}

\date{
\begin{tabular}{ll}
  {Orientadores:} & Pedro Matutino, e-mail: pedro.miguens@isel.pt \\
                %  & Alberto Caeiro, e-mail: ac@pc.com, PersonCompany\\
\end{tabular}\\
\vspace{5mm}
Mar\c{c}o de 2025}

\begin{document}

\maketitle

\section*{Introdução}
Nesta secção aborda-se o enquadramento do projeto e a sua finalidade. O logótipo do ISEL é apresentado na figura~\ref{fig:logotipo}. Esta proposta está organizada em três secções...

\begin{figure}[h]
\begin{center}
\resizebox{80mm}{!}{\includegraphics{logoISEL.png}}
\end{center}
\caption{Legenda da figura com o logótipo do ISEL.}\label{fig:logotipo}
\end{figure}

O livro \cite{dawson2015projects} é um guia para estudantes... 

Na Wikipedia, o verbete \cite{enwiki:1274363095} é sobre \LaTeX. Em \cite{6547630} ...


Exemplo de equação
\begin{equation}
    E_0 = mc^2                         
\end{equation}

A tabela~\ref{tab:ex1} é um exemplo de tabela.

\begin{table}
\begin{center}
\begin{tabular}{ccc}
	\hline
	Nome & Tarefas & Progresso \\
	\hline
	Ricardo Reis & 1 e 2 & 50\% \\
	Álvaro Campos & 1 e 3 & 20\%  \\
	\hline
\end{tabular}
\caption{Exemplo de tabela.} \label{tab:ex1}
\end{center}
\end{table}

\section*{Análise}
Nesta secção...

\section*{Planeamento}
Agora o texto sobre o planeamento

\bibliographystyle{unsrt}
\bibliography{referencias}

\end{document}